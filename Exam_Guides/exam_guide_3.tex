% vim: nu expandtab shiftwidth=2 softtabstop=2 foldmethod=marker

\lstset{language=C++,
	frame=tb,
	tabsize=2,
	showstringspaces=false,
	commentstyle=\color{green},
	keywordstyle=\color{blue},
	stringstyle =\color{red}}

\section{Final Exam Guide}

\subsection{Mathematical Tools}
  \begin{itemize}
  \item Definition and properties of floor, ceiling and logarithmic functions
    \begin{enumerate}
    \item floor function
      \begin{itemize}
      \item $\left\lfloor x\right\rfloor$ is an integer
      \item $\left\lfloor x\right\rfloor < x < \left\lfloor x\right\rfloor+1$
      \item $\left\lfloor\sqrt{x}\right\rfloor < \sqrt{x} < \left\lfloor\sqrt{x}\right\rfloor+1$
      \end{itemize}
    \item ceiling function
      \begin{itemize}
      \item $\left\lceil x \right\rceil$ is an integer
      \item $\left\lceil x \right\rceil > x > \left\lceil x \right\rceil-1$
      \item $\left\lceil\sqrt{x}\right\rceil > \sqrt{x} > \left\lceil\sqrt{x}\right\rceil-1$
      \end{itemize}
    \end{enumerate}
  \item Definition and properties of asymptotic notations, focus on: Big-Oh, Big-Omega, and Big-Theta
    \begin{enumerate}
    \item Big-Oh
      \begin{itemize}
      \item $f(n)=O(g(n)) \leftrightarrow$ backwards e constants c, $n_{0}$ $f(n) \leq cg(n)$ for all $n\geq n_{0}$
      \item $\left\lceil\sqrt{n}\right\rceil=O(\sqrt{n})$ because $\left\lceil\sqrt{n}\right\rceil\leq\sqrt{n}+1\leq 2\sqrt{n}$ for all $n\geq 1$
      \end{itemize}
    \item Big-Omega
      \begin{itemize}
      \item $f(n)=Ohmega(g(n)) \leftrightarrow$ backwards e constants c, $n_{0}$ $f(n) \leq cg(n)$ for all $n\geq n_{0}$
      \item $\left\lceil\sqrt{n}\right\rceil=Ohmega(\sqrt{n})$ because $\left\lceil\sqrt{n}\right\rceil\leq\sqrt{n}+1$ for all $n\geq 1$
      \item Then $\left\lceil\sqrt{n}\right\rceil=\Theta(\sqrt{n})$
      \end{itemize}
    \item Big-Theta
    \end{enumerate}
  \item Recognize linear recurrence relations and its components (coefficients, nonhomogeneous terms)
  \item Methods to solve recurrence relations (characteristic equation, backward substitution). How and when to use those methods?
  \item Definition and properties of binary trees.
    \begin{itemize}
    \item Height of a tree with n nodes
      \begin{itemize}
      \item minimum $\left\lfloor\log{(n)}\right\rfloor$ (when the tree is balanced)
      \item maximum $n-1$ (when it is in a straight line
      \end{itemize}
    \item Number of empty subtrees in a tree with n nodes is n+1
    \item Number of leaves in a tree with n nodes is
      \begin{itemize}
      \item minimum 1 leaf
      \item maximum $\left\lfloor\frac{n}{2}\right\rfloor$
      \end{itemize}
    \end{itemize}
  \end{itemize}
\subsection{Algorithm Analysis}
  \begin{itemize}
  \item How to count the number of iterations of a single loop, and  nested loops? (See sample questions in page 2 below)
  \item How to analyze worst case space complexity of a function call
    \begin{itemize}
    \item To analyze space complexity, Two things you need to mention are
      \begin{enumerate}
      \item [1] The size of each stack frame.
      \item [2] The maximum number of stack frames on the runtime stack
      \end{enumerate}
    \end{itemize}
  \item How to analyze the best case \& worst case time complexities of an algorithm
    \begin{itemize}
    \item [***Note***] this is a focus of the exam, use the below as an example
    \item Best Case time complexity $T(n)$ of an algorithm is the minimum running of that algorithm on any input of size n
    \item Worst Case time complexity $T(n)$ of an algorithm is the maximum running of that algorithm on any input of size n
    \end{itemize}
  \item Understand the deterministic analysis of the following algorithms:
    \begin{itemize}
    \item Euclid's GCD
    \item linear search
    \item binary search
    \item merge--sort
    \item quicksort
    \item binary tree traversal (print in order)
    \item searching a binary search tree
    \item insert a new node into a binary search tree
    \end{itemize}
  \item Understand the theorem for the lower bound of comparison-based sorting algorithms (Theorem 5.9.1 in the textbook)
    \begin{itemize}
    \item [***Note***] not a focus of the exam
    \item Need to understand what it implies in the slides
    \item Exercise $5.9.5$ in the textbook pages $278-279$ the answer is no for a because they use the word always.
    \item The theory is only for the lower bound on the worst case time complexity of any comparison based sorting algorithm
    \end{itemize}
  \item Understand the probabilistic analysis of the following algorithms:
    \begin{itemize}
    \item [***NOTE***] only one big thing you need to remember is that in order to perform average case analysis for any algorithm.
    \item I was correct that you use induction for this if you want.
      \begin{itemize}
      \item Assume the input is random
      \item Estimate the expected running time of the algorithm
      \end{itemize}
    \item linear search
    \item quicksort
    \end{itemize}
  \end{itemize}
\subsection{sample questions}
\begin{enumerate}
\item How many times does the word ``Hi'' is printed out by the following loop, assuming variable n has been declared and given an arbitrary positive integer?
  \begin{lstlisting}
for (i=1; i*i<=n; i++)
  cout << "Hi";
  \end{lstlisting}
  \begin{enumerate}
  \item n
  \item $\left\lfloor\sqrt{n}\right\rfloor$
  \item $n\left\lfloor\sqrt{n}\right\rfloor$
  \item $n^{2}$
  \end{enumerate}
\item How many times does the word ``Hi'' is printed out by the following loop, assuming variable n has been declared and given an arbitrary positive integer?
  \begin{lstlisting}
for (i=1, k=n; i<n; i++, k=k/2)
  for (j=0; j<k; j++)
    cout << "Hi";
  \end{lstlisting}
  \begin{enumerate}
  \item $\sum\limits_{i=0}^{n-1}\left\lfloor\frac{n}{2^{i}}\right\rfloor$
  \item n
  \item $\frac{n}{2}$
  \item $n^{2}$
  \end{enumerate}
\end{enumerate}
