% vim: nu expandtab shiftwidth=2 softtabstop=2 foldmethod=marker

\definecolor{LightCyan}{rgb}{0,1,1}
\definecolor{DarkCyan}{rgb}{0,0.88,0.88}

\lstset{language=C++,
	frame=tbl,
	tabsize=2,
	showstringspaces=false,
	commentstyle=\color{green},
	keywordstyle=\color{blue},
	stringstyle =\color{red}}

\subsection{Euclidean Algorithm for GCD}
\item Computing Greatest Common Divisors (GCD)
  \begin{itemize}
  \item The \textcolor{DarkCyan}{greatest common divisor} of two integers a and b
    \begin{itemize}
    \item denoted \textcolor{DarkCyan}{gcd(a,b)}
    \item is the largest positive integer that divides both a and b
    \end{itemize}
  \item Euclid's algorithm for computing GCD
    \begin{itemize}
    \item One of the oldest algorithms in common use $(>2000$ years)
    \item An important routine in the RSA encryption
    \item Based on \textbf{Theorem:} Let m and n be positive integers, with $m\leq n$. Then $gcd(m,n)=gcd(n mod m,m)$
      \begin{itemize}
      \item Example: $gcd(260,412)=gcd(152,260)$ because $412\%260=152$
      \end{itemize}
    \end{itemize}
  \end{itemize}
\item Euclid's GCD algorithm \\
\noindent\begin{minipage}[t]{0.5\textwidth} %left side
\textbf{Recursive Version}
\begin{lstlisting}
int gcd(int m, int n)
//assume 0<=m<=n
{
  if (m==0)
    return n;
  else
    return gcd(n%m, m);
}
\end{lstlisting}
\end{minipage}
\begin{minipage}[t]{0.5\textwidth} %right side
\textbf{Iterative Version}
\begin{lstlisting}
int gcd(int m, int n)
//assume 0<=m<=n
{
  int dividend=n;
  int divisor=m;
  while (divisor !=0)
  {
    int remainder=dividend%divisor;
    dividend=divisor;
    divisor=remainder;
  }
  return dividend;
}
\end{lstlisting}
\end{minipage}
\item Analyzing Euclid's Algorithm
  \begin{itemize}
  \item Consider he iterative version
  \item Will focus on worst case time complexity
  \item The number of iterations of the \textcolor{DarkCyan}{while} loop
    \begin{itemize}
    \item Equals the number of divisions executed in the lop
    \item Example: input $n=412$, $m=260\rightarrow6$ divisions $\rightarrow6$ iterations
    \end{itemize}
  \end{itemize}
  \begin{tabular}{l r r r r r r r}
  \rowcolor{LightCyan} \textbf{Division\#} & 1   & 2   & 3   & 4   & 5  & 6  & End of loop \\
  \rowcolor{DarkCyan} dividend            & 412 & 260 & 152 & 108 & 44 & 20 & 4 \\
  \rowcolor{LightCyan} divisor            & 260 & 152 & 108 & 44  & 20 & 4  & 0 \\
  \rowcolor{DarkCyan} remainder           & 152 & 108 & 44  & 20  & 4  & 0  &   \\
  \end{tabular}
\item Formal Analysis of Euclid's Algorithm
  \begin{itemize}
  \item Consider the Euclid algorithm on inputs \emph{m} and \emph{n}
    \begin{itemize}
    \item Assume $0\leq m\leq n$
    \item Let $k$ denote the total number of divisions or iterations
    \item Let $d_{i}$ be the divisor of the $i^{th}$ division from the last one.
    \item For all $i=1$,...,$k-1$, $d_{i}=d_{i+2}$ mod $d_{i+1}$
    \item [$\Rightarrow$] $d_{i} < d_{i+1}$ and $d_{i+1}=\left\lfloor\frac{d_{i+2}}{d_{i+1}}\right\rfloor d_{i+1}+d_{i}$
    \end{itemize}
  \end{itemize}
  \begin{tabular}{l c c c c c c}
  \rowcolor{LightCyan} \textbf{Division\#} & 1   & 2   & ...   & k-2   & k-1  & k (Last) \\
  \rowcolor{DarkCyan} dividend            & $d_{k+1}=n$ & $d_{k}$ & ... & $d_{4}$ & $d_{3}$ & $d_{2}$ \\
  \rowcolor{LightCyan} divisor            & $d_{k}=m$ & $d_{k-1}$ & ... & $d_{3}$  & $d_{2}$ & $d_{1}$ \\
  \rowcolor{DarkCyan} remainder           & $d_{k-1}$ & $d_{k-2}$ & ...  & $d_{2}$  & $d_{1}$  & 0 \\
  \end{tabular}
\item Formal Analysis of Euclid's Algorithm
  \begin{itemize}
  \item Now find a \emph{good} lower bound on $d_{i}$'s
    \begin{itemize}
    \item $d_{1}\geq 1$   since a valid divisor must be nonzero
    \item $d_{2}\geq 2$   since $d_{2}>d_{1}$ \& $d_{i}$'s are integers
    \item $d_{i+2}\geq d_{i+1} +d_{i}$ since $d_{i+2}=\left\lfloor\frac{d_{i+2}}{d_{i+1}}\right\rfloor d_{i+1}+d_{i}$ and $\frac{d_{i+2}}{d_{i+1}}\geq 1$
    \end{itemize}
  \end{itemize}
  \begin{tabular}{l c c c c c c c}
  \rowcolor{LightCyan} \textbf{Division\#} & ... & $k-5$ & $k-4$ & $k-3$ & $k-2$  & $k-1$ & k (Last) \\
  \rowcolor{DarkCyan} dividend & & $d_{7}$ & $d_{6}$ & $d_{5}$ & $d_{4}$ & $d_{3}$ & $d_{2}$ \\
  \rowcolor{LightCyan} divisor & & $d_{6}$ & $d_{5}$ & $d_{4}$ & $d_{3}$ & $d_{2}$ & $d_{1}$ \\
  \rowcolor{DarkCyan} divisor $>=$ & & 13 & 8 & 5 & 3 & 2 & 1 \\
  \rowcolor{LightCyan} remainder & & $d_{5}$ & $d_{4}$ & $d_{3}$ & $d_{2}$ & $d_{1}$ & 0 \\
  \end{tabular}
\item Formal Analysis of Euclid's Algorithm
  \begin{itemize}
  \item Can prove by induction: $d_{i}\geq F_{i+1}$ for all $i\geq 1$
    \begin{itemize}
    \item Where $F_{i}$ is the $i^{th}$ Fibonacci number
    \end{itemize}
  \item Since $d_{k}=m$, we have 
  \item [] $m\geq F_{k+1}=\frac{1}{\sqrt{5}}\left(\frac{1+\sqrt{5}}{2}\right)^{k+1} -\frac{1}{\sqrt{5}}\left(\frac{1-\sqrt{5}}{2}\right)^{k+1}$ 
  \item [] $\Rightarrow m\geq\frac{1}{\sqrt{5}}\left(\frac{1+\sqrt{5}}{2}\right)^{k+1}-1$ 
  \item [] $\Rightarrow k=O(\log{(m)})$
  \item Worst case time complexity: O(log(min\{m,n\}))
  \end{itemize}
\item {\color{green}More Examples/Practice}
  \begin{itemize}
  \item What is the worst case time complexity of the recursive version of the Euclid’s algorithm?
  \end{itemize}
