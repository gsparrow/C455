% vim: nu expandtab shiftwidth=2 softtabstop=2 foldmethod=marker

\documentclass{article}
\usepackage{fancyhdr}
\def\mydate{\leavevmode\hbox{\the\year-\twodigits\month-\twodigits\day}}
\def\twodigits#1{\ifnum#1<10 0\fi\the#1}
\title{Homework 3}
\date{\mydate}
\author{George Sparrow}

\fancypagestyle{plain}
{
  \fancyhf{}
  \lfoot{\author{George Sparrow}}
  \cfoot{\thepage}
  \rfoot{\date{\mydate}}
  \renewcommand{\headrulewidth}{0pt}
}
\pagestyle{plain}

\begin{document}
\maketitle
\newpage

\begin{enumerate}
\item Find the general solution in closed form to each recurrence relation. For full credits, you need to justify your answers (if you apply some theorem, you need to cite them).
  \begin{enumerate}
  \item $A_{n}=3A_{n-1}+2^{n}$
    \begin{itemize} %{{{
    \item Solving using the general Solution Formula
      \begin{itemize} %{{{
      \item The formula is $$F_{n}=c^{n}F_{0} + \sum\limits_{i=0}^{n-1}c^{i}f(n-i)$$
      \item In this problem, $c=3$ and $f(n)=2^{n}$
      \item Due to $f(n)=2^{n}$ being an exponential equation, defined by ($f(n)=b^{n}$ where $n\neq0$), we modify the formula. In this case, $b=2$, as $c\neq b$, we modify the formula to be $$F_{n}=c^{n}F_{0} + b^{n}\left(\frac{(\frac{c}{b})^{n}-1}{(\frac{c}{b})-1}\right)$$
      \item Plugging everything into the formula, we get $$A_{n}=3^{n}A_{0} + 2^{n}\left(\frac{(\frac{3}{2})^{n}-1}{(\frac{3}{2})-1}\right)$$
      \item [*] In order to prove that this $$A_{n}=3^{n}A_{0} + 2^{n}\left(\frac{(\frac{3}{2})^{n}-1}{(\frac{3}{2})-1}\right)$$ is a solution to the nonhomogeneous recurrence relation ($A_{n}=3A_{n-1}+2^{n}$), we substitute it (into $3A_{n-1}+2^{n}$) in order to attempt to get an identity $$A_{n}=3^{n}A_{0} + 2^{n}\left(\frac{(\frac{3}{2})^{n}-1}{(\frac{3}{2})-1}\right)$$
      \item [*] $3\left(3^{n-1}A_{0} + 2^{n-1}\left(\frac{(\frac{3}{2})^{n-1}-1}{(\frac{3}{2})-1}\right)\right)+2^{n}$
      \item [*] $3^{n}A_{0} + 3*2^{n-1}\left(\frac{(\frac{3}{2})^{n-1}-1}{(\frac{3}{2})-1}\right)+2^{n}$
      \item [*] $3^{n}A_{0} + \frac{2^{n}}{2}*3\left(\frac{(\frac{3}{2})^{n-1}-1}{(\frac{3}{2})-1}\right)+2^{n}$
      \item [*] $3^{n}A_{0} + 2^{n} \left(\frac{1}{2}*3\left(\frac{(\frac{3}{2})^{n-1}-1}{(\frac{3}{2})-1}\right)+1\right)$
      \item [*] $3^{n}A_{0} + 2^{n} \left(\frac{1}{2}*3\left(\frac{(\frac{3}{2})^{n-1}-1}{(\frac{1}{2})}\right)+1\right)$
      \item [*] $3^{n}A_{0} + 2^{n} \left(3\left(\frac{2*3^{n}}{3*2^{n}}-1\right)+1\right)$
      \item [*] $3^{n}A_{0} + 2^{n} \left(3\left(\frac{2*3^{n}}{3*2^{n}}-\frac{3*2^{n}}{3*2^{n}}\right)+1\right)$
      \item [*] $3^{n}A_{0} + 2^{n} \left(3\left(\frac{2*3^{n}-3*2^{n}}{3*2^{n}}\right)+1\right)$
      \item [*] $3^{n}A_{0} + 2^{n} \left(\frac{2*3^{n}-3*2^{n}}{2^{n}}+1\right)$
      \item [*] $3^{n}A_{0} + 2*3^{n}-3*2^{n}+2^{n}$
      \item [*] $3^{n}A_{0} + 2*3^{n}-2*2^{n}$
      \item [*] $3^{n}A_{0} + 2*3^{n}-2^{n+1}$
      \item [*] $3^{n}A_{0} + 2(3^{n}-2^{n})$
      \item [*] $3^{n}A_{0} + 2*2^{n}\left(\frac{3^{n}-2^{n}}{2^{n}}\right)$
      \item [*] $3^{n}A_{0} + 2^{n}\left(\frac{\frac{3^{n}-2^{n}}{2^{n} }}{\frac{1}{2}}\right)$
      \item [*] $3^{n}A_{0} + 2^{n}\left(\frac{\frac{3^{n}}{2^{n}}-1}{\frac{1}{2}}\right)$
      \item [*] $3^{n}A_{0} + 2^{n}\left(\frac{\left(\frac{3}{2}\right)^{n}-1}{\frac{1}{2}}\right)$
      \item [*] $3^{n}A_{0} + 2^{n}\left(\frac{\left(\frac{3}{2}\right)^{n}-1}{\left(\frac{3}{2}\right)-1}\right)$
      \item [*] Since we have the identity $$3^{n}A_{0} + 2^{n}\left(\frac{\left(\frac{3}{2}\right)^{n}-1}{\left(\frac{3}{2}\right)-1}\right)$$, $$3^{n}A_{0} + 2^{n}\left(\frac{\left(\frac{3}{2}\right)^{n}-1}{\left(\frac{3}{2}\right)-1}\right)$$ is a solution to $A_{n}=3A_{n-1}+2^{n}$
      \end{itemize} %}}}
    \item Solving using Parametization of Variables ($f(n)=g(n)+h(n)$). This is the mechanism that allows us to split more complex nonhomogeneous equations into multiple simpler nonhomogeneous equations.
      \begin{itemize} %{{{
      \item The first step is to solve the linear homogeneous relation portion of this equation.
      \item $A_{n}=3A_{n-1}$
      \item [] All things in the form $F_{n}=cF_{n-1}$ for all $n\geq1$ where $c_{i}$ is a nonzero constant, are homogeneous linear recurrrence relations.
      \item [] The general solution to all first order linear homogenous recurrence relations is $F_{n}=F_{0}c^{n}$
      \item Given that c=3, and $F_{0}=A_{0}$ is not defined, the solution to the linear homogeneous relation is $A_{0}3^{n}$
      \item [*] In order to prove that this ($A_{0}3^{n}$) is a solution to the recurrence relation ($3A_{n-1}$), we substitute it (into $3A_{n-1}$) in order to attempt to get an identity ($A_{0}3^{n}$).
      \item [*] $3*(A_{0}3^{n-1})$
      \item [*] $A_{0}3^{n}$
      \item [*] Since we have the identity $A_{0}3^{n}$, $A_{0}3^{n}$ is a solution to $3A_{n-1}$
      \item Now we solve the homogeneous part. Let $A_{n}=c2^{n}$ 
      \item Since we have $A_{n}=c2^{n}$ we also have $A_{n-1}=c2^{n-1}$
      \item Now we substitue $A_{n-1}=c2^{n-1}$ and $A_{n}=c2^{n}$ into $A_{n}=3A_{n-1}+2^{n}$ in order to solve for c.
      \item $c2^{n}=3(c2^{n-1})+2^{n}$
      \item $c2^{n}=3c2^{n-1}+2^{n}$
      \item $c2^{n}-2^{n}=3c2^{n-1}$
      \item $2^{n}(c-1)=3c2^{n-1}$
      \item $2^{n}(c-1)=\frac{3c2^{n}}{2}$
      \item $(c-1)=\frac{3c}{2}$
      \item $2(c-1)=3c$
      \item $2c-2=3c$
      \item $2c=3c+2$
      \item $-1c=2$
      \item $c=-2$
      \item Since $c=-2$, this gives us $A_{n}=-2(2^{n})$ which gives us $A_{n}=-2^{n+1}$
      \item Now, we put it all together.
      \item $A_{0}3^{n}-2^{n+1}$
      \item [*] In order to prove that this ($A_{0}3^{n}-2^{n+1}$) is a solution to the nonhomogeneous recurrence relation ($A_{n}=3A_{n-1}+2^{n}$), we substitute it (into $3A_{n-1}+2^{n}$) in order to attempt to get an identity ($A_{0}3^{n}-2^{n+1}$)
      \item [*] $3(A_{0}3^{n-1}-2^{n-1+1})+2^{n}$
      \item [*] $3A_{0}3^{n-1}-3*2^{n-1+1}+2^{n}$
      \item [*] $3A_{0}3^{n-1}-3*2^{n}+2^{n}$
      \item [*] $A_{0}3^{n}-3*2^{n}+2^{n}$
      \item [*] $A_{0}3^{n}-2*2^{n}$
      \item [*] $A_{0}3^{n}-2^{n+1}$
      \item [*] Since we have the identity $A_{0}3^{n}-2^{n+1}$, $A_{0}3^{n}-2^{n+1}$ is a solution to $A_{n}=3A_{n-1}+2^{n}$
      \end{itemize} % }}}
    \end{itemize} %}}}
  \item $A_{n}=3A_{n-1}+3^{n}$
    \begin{itemize}
    \item Solving using the general Solution Formula
      \begin{itemize} % {{{
      \item The formula is $$F_{n}=c^{n}F_{0} + \sum\limits_{i=0}^{n-1}c^{i}f(n-i)$$
      \item In this problem, $c=3$ and $f(n)=3^{n}$
      \item Due to $f(n)=3^{n}$ being an exponential equation, defined by ($f(n)=b^{n}$ where $n\neq0$), we modify the formula. In this case, $b=3$, as $c= b$, we modify the formula to be $$F_{n}=c^{n}F_{0} + nb^{n}$$
      \item Plugging in to the formula, we get $A_{n}=3^{n}A_{0} + n3^{n}$
      \item [*] In order to prove that this ($A_{0}3^{n}+n3^{n}$) is a solution to the nonhomogeneous recurrence relation ($A_{n}=3A_{n-1}+3^{n}$), we substitute it (into $3A_{n-1}+3^{n}$) in order to attempt to get an identity ($A_{0}3^{n}+n3^{n}$)
      \item [*] $3(A_{0}3^{n-1} + (n-1)3^{n-1}) + 3^{n}$
      \item [*] $3A_{0}3^{n-1} + 3(n-1)3^{n-1} + 3^{n}$
      \item [*] $A_{0}3^{n} + (n-1)3^{n} + 3^{n}$
      \item [*] $A_{0}3^{n} + 3^{n}((n-1) + 1)$
      \item [*] $A_{0}3^{n} + 3^{n}(n-1+1)$
      \item [*] $A_{0}3^{n} + 3^{n}n$
      \item [*] $A_{0}3^{n} + n3^{n}$
      \item [*] Since we have the identity $A_{0}3^{n}+n3^{n}$, $A_{0}3^{n}+n3^{n}$ is a solution to $A_{n}=3A_{n-1}+3^{n}$
      \end{itemize} %}}}
    \item Solving using Parametization of Variables ($f(n)=g(n)+h(n)$). This is the mechanism that allows us to split more complex nonhomogeneous equations into multiple simpler nonhomogeneous equations.
      \begin{itemize} %{{{
      \item The first step is to solve the linear homogeneous relation portion of this equation.
      \item $A_{n}=3A_{n-1}$
      \item [] All things in the form $F_{n}=cF_{n-1}$ for all $n\geq1$ where $c_{i}$ is a nonzero constant, are homogeneous linear recurrrence relations.
      \item [] The general solution to all first order linear homogenous recurrence relations is $F_{n}=F_{0}c^{n}$
      \item Given that c=3, and $F_{0}=A_{0}$ is not defined, the solution to the linear homogeneous relation is $A_{0}3^{n}$
      \item [*] In order to prove that this ($A_{0}3^{n}$) is a solution to the recurrence relation ($3A_{n-1}$), we substitute it (into $3A_{n-1}$) in order to attempt to get an identity ($A_{0}3^{n}$).
      \item [*] $3*(A_{0}3^{n-1})$
      \item [*] $A_{0}3^{n}$
      \item [*] Since we have the identity $A_{0}3^{n}$, $A_{0}3^{n}$ is a solution to $3A_{n-1}$
      \item Now we solve the homogeneous part. Normally, we would let $A_{n}=c3^{n}$, however, in this case the base ($3$) is equal to the constant multiplier in the homogeneous portion ($3$), and so we use the formula provided in the notes that says when $c=b$ is true, then $nb^{n}$ is the solution. So the solution to the non-homogeneous part is $n3^{n}$
      \item Now, we put it all together.
      \item $A_{n}=A_{0}3^{n} + n3^{n}$
      \item [*] In order to prove that this ($A_{0}3^{n}+n3^{n}$) is a solution to the nonhomogeneous recurrence relation ($A_{n}=3A_{n-1}+3^{n}$), we substitute it (into $3A_{n-1}+3^{n}$) in order to attempt to get an identity ($A_{0}3^{n}+n3^{n}$)
      \item [*] $3(A_{0}3^{n-1} + (n-1)3^{n-1}) + 3^{n}$
      \item [*] $3A_{0}3^{n-1} + 3(n-1)3^{n-1} + 3^{n}$
      \item [*] $A_{0}3^{n} + (n-1)3^{n} + 3^{n}$
      \item [*] $A_{0}3^{n} + 3^{n}((n-1) + 1)$
      \item [*] $A_{0}3^{n} + 3^{n}(n-1+1)$
      \item [*] $A_{0}3^{n} + 3^{n}n$
      \item [*] $A_{0}3^{n} + n3^{n}$
      \item [*] Since we have the identity $A_{0}3^{n}+n3^{n}$, $A_{0}3^{n}+n3^{n}$ is a solution to $A_{n}=3A_{n-1}+3^{n}$
      \end{itemize} %}}}
    \end{itemize}
  \item $A_{n}=3A_{n-1}-2^{n}+10*3^{n}$
    \begin{itemize} %{{{
    \item Solving using Parametization of Variables ($f(n)=g(n)+h(n)$). This is the mechanism that allows us to split more complex nonhomogeneous equations into multiple simpler nonhomogeneous equations.
    \item The first step is to solve the linear homogeneous relation portion of this equation.
    \item [] All things in the form $F_{n}=cF_{n-1}$ for all $n\geq1$ where $c_{i}$ is a nonzero constant, are homogeneous linear recurrrence relations.
    \item [] The general solution to all first order linear homogenous recurrence relations is $F_{n}=F_{0}c^{n}$
    \item $A_{n}=3A_{n-1}$
    \item Given that c=3, and $A_{0}$ is not defined, the solution to the linear homogeneous relation is $A_{0}3^{n}$
    \item [*] In order to prove that this ($A_{0}3^{n}$) is a solution to the recurrence relation ($3A_{n-1}$), we substitute it (into $3A_{n-1}$) in order to attempt to get an identity ($A_{0}3^{n}$).
    \item [*] $3*(A_{0}3^{n-1})$
    \item [*] $A_{0}3^{n}$
    \item [*] Since we have the identity $A_{0}3^{n}$, $A_{0}3^{n}$ is a solution to $3A_{n-1}$
    \item Her method using the formula
      \begin{itemize} %{{{
      \item Now we solve the simpler homogeneous parts. First we solve $A_{n}=3A_{n-1}-2^{n}$
      \item The formula is $$F_{n}=c^{n}F_{0} + \sum\limits_{i=0}^{n-1}c^{i}f(n-i)$$
      \item In this problem, $c=3$ and $f(n)=-2^{n}$
      \item Due to $f(n)=2^{n}$ being an exponential equation, defined by ($f(n)=b^{n}$ where $n\neq0$), we modify the formula. In this case, $b=-2$, as $c\neq b$, we modify the formula to be $$F_{n}=c^{n}F_{0} + b^{n}\left(\frac{(\frac{c}{b})^{n}-1}{(\frac{c}{b})-1}\right)$$
      \item Plugging everything into the formula, we get $$A_{n}=3^{n}A_{0} -2^{n}\left(\frac{(\frac{3}{-2})^{n}-1}{(\frac{3}{-2})-1}\right)$$
      \item [*] In order to prove that this $$A_{n}=3^{n}A_{0} -2^{n}\left(\frac{(\frac{3}{-2})^{n}-1}{(\frac{3}{-2})-1}\right)$$ is a solution to the nonhomogeneous recurrence relation ($A_{n}=3A_{n-1}+2^{n}$), we substitute it (into $3A_{n-1}+2^{n}$) in order to attempt to get an identity $$A_{n}=3^{n}A_{0} -2^{n}\left(\frac{(\frac{3}{-2})^{n}-1}{(\frac{3}{-2})-1}\right)$$
      \item [*] $3\left(3^{n-1}A_{0} -2^{n-1}\left(\frac{(\frac{3}{-2})^{n-1}-1}{(\frac{3}{-2})-1}\right)\right)-2^{n}$
      \item [*] $3^{n}A_{0} -3*2^{n-1}\left(\frac{(\frac{3}{-2})^{n-1}-1}{(\frac{3}{-2})-1}\right)-2^{n}$
      \item [*] $3^{n}A_{0} -3*2^{n-1}\left(\frac{(\frac{3}{-2})^{n-1}-1}{(\frac{5}{-2})}\right)-2^{n}$
      \item [*] $3^{n}A_{0} -3*2^{n-1}\left(\frac{\frac{3^{n-1}-2^{n-1}}{-2^{n-1} }}{\frac{5}{-2}}\right)-2^{n}$
      \item [*] $3^{n}A_{0} -3*2^{n-1}\left(\frac{(3^{n-1}-2^{n-1})*(-2)}{-2^{n-1}*(5) }\right)-2^{n}$
      \item [*] $3^{n}A_{0} +3*\left(\frac{(3^{n-1}-2^{n-1})*(-2)}{5}\right)-2^{n}$
      \item [*] $3^{n}A_{0} +3*\left(\frac{-2*3^{n-1}+2^{n}}{5}\right)-2^{n}$
      \item [*] $3^{n}A_{0} +\frac{-2*3^{n}}{5} +\frac{3*2^{n}}{5}-2^{n}$
      \item [*] $3^{n}A_{0} +\frac{-2*3^{n}}{5} +\frac{3}{5}*2^{n}-2^{n}$
      \item [*] $3^{n}A_{0} +\frac{-2*3^{n}}{5} -\frac{2}{5}*2^{n}$
      \item [*] $3^{n}A_{0} -\frac{2}{5}*3^{n} -\frac{2}{5}*2^{n}$
      \item [*] $3^{n}A_{0} -\frac{2}{5}\left(3^{n}+2^{n}\right)$
      \item [*] $3^{n}A_{0} -2^{n}(-\frac{2}{5})\left(\frac{3^{n}+2^{n}}{-2^{n}}\right)$
      \item [*] $3^{n}A_{0} -2^{n}(-\frac{2}{5})\left(\frac{3^{n}}{-2^{n}}-1\right)$
      \item [*] $3^{n}A_{0} -2^{n}(-\frac{2}{5})\left(\left(\frac{3}{-2}\right)^{n}-1\right)$
      \item [*] $3^{n}A_{0} -2^{n}\left(\frac{\left(\frac{3}{-2}\right)^{n}-1}{\frac{5}{-2}}\right)$
      \item [*] $3^{n}A_{0} -2^{n}\left(\frac{\left(\frac{3}{-2}\right)^{n}-1}{\left(\frac{3}{-2}\right)-1}\right)$
      \item [*] Since we have the identity $$3^{n}A_{0} -2^{n}\left(\frac{\left(\frac{3}{-2}\right)^{n}-1}{\left(\frac{3}{-2}\right)-1}\right)$$, $$3^{n}A_{0} -2^{n}\left(\frac{\left(\frac{3}{-2}\right)^{n}-1}{\left(\frac{3}{-2}\right)-1}\right)$$ is a solution to $A_{n}=3A_{n-1}-2^{n}$
      \end{itemize} %}}}
    \item My Method using Parametization of Variables
      \begin{itemize} %{{{
      \item Now we would normally solve the entire nonhomogenous part, but as this is a complex nonomogeneous part, we can solve each part of it with the homogeneous part in turn. So first now we solve the nonhomogenous relation that is $A_{n}=3A_{n-1}-2^{n}$, and since we have already solved the homogeneous part of it, now we just solve $-2^{n}$. Let $A_{n}=-c2^{n}$
      \item Since we have $A_{n}=-c2^{n}$ we also have $A_{n-1}=-c2^{n-1}$
      \item Now we substitue $A_{n-1}=-c2^{n-1}$ and $A_{n}=-c2^{n}$ into $A_{n}=3A_{n-1}-2^{n}$ in order to solve for c.
      \item $-c2^{n}=3(-c2^{n-1})-2^{n}$
      \item $-c2^{n}+2^{n}=-3c2^{n-1}$
      \item $2^{n}(-c+1)=-3c2^{n-1}$
      \item $2^{n}(-c+1)=-3c2^{-1}2^{n}$
      \item $-c+1=-3c2^{-1}$
      \item $-c+1=-\frac{3}{2}c$
      \item $\frac{1}{2}c+1=0$
      \item $\frac{1}{2}c=-1$
      \item $c=-2$
      \item Since $c=-2$, this gives us $A_{n}=-2(-2^{n})$ which gives us $A_{n}=2^{n+1}$
      \item Now, we put it together.
      \item $A_{n}=A_{0}3^{n}+2^{n+1}$
      \item [*] In order to prove that this ($A_{0}3^{n}+2^{n+1}$) is a solution to the nonhomogeneous recurrence relation ($A_{n}=3A_{n-1}-2^{n}$), we substitute it (into $3A_{n-1}-2^{n}$) in order to attempt to get an identity ($A_{0}3^{n}+2^{n+1}$)
      \item [*] $3(A_{0}3^{n-1}+2^{n})-2^{n}$
      \item [*] $3A_{0}3^{n-1}+3*2^{n}-2^{n}$
      \item [*] $A_{0}3^{n}+3*2^{n}-2^{n}$
      \item [*] $A_{0}3^{n}+2*2^{n}$
      \item [*] $A_{0}3^{n}+2^{n+1}$
      \item [*] Since we have the identity $A_{0}3^{n}+2^{n+1}$, $A_{0}3^{n}+2^{n+1}$ is a solution to $A_{n}=3A_{n-1}-2^{n}$
      \end{itemize} %}}}
    \item Her method using the formula
      \begin{itemize} %{{{
      \item Now we solve the other simpler homogeneous part. First we solve $A_{n}=3A_{n-1}+10*3^{n}$
      \item The formula is $$F_{n}=c^{n}F_{0} + \sum\limits_{i=0}^{n-1}c^{i}f(n-i)$$
      \item In this problem, $c=3$ and $f(n)=10*3^{n}$
      \item Due to $f(n)=2^{n}$ being an exponential equation, defined by ($f(n)=b^{n}$ where $n\neq0$), we modify the formula. In this case, $b=3$, as $c=b$, we modify the formula to be $$F_{n}=c^{n}F_{0} + nb^{n}$$
      \item Plugging everything into the formula, we get $$A_{n}=3^{n}A_{0}+10n*3^{n}$$
      \item [*] In order to prove that this $$A_{n}=3^{n}A_{0}+10n*3^{n}$$ is a solution to the nonhomogeneous recurrence relation ($A_{n}=3A_{n-1}+10*3^{n}$), we substitute it (into $3A_{n-1}+10*3^{n}$) in order to attempt to get an identity $$A_{n}=3^{n}A_{0}+10n*3^{n}$$
      \item [*] $3(3^{n-1}A_{0}+10(n-1)3^{n-1})+10*3^{n}$
      \item [*] $3^{n}A_{0}+10(n-1)3^{n}+10*3^{n}$
      \item [*] $3^{n}A_{0}+10n*3^{n}-10*3^{n}+10*3^{n}$
      \item [*] $3^{n}A_{0}+10n*3^{n}$
      \item [*] Since we have the identity $3^{n}A_{0}+10n3^{n}$, $3^{n}A_{0}+10n3^{n}$ is a solution to $A_{n}=3A_{n-1}+10*3^{n}$
      \end{itemize} %}}}
    \item My Method using Parametization of Variables
      \begin{itemize} %{{{
      \item Now since we solved one nonhomogeneous part $-2^{n}$, now we solve the other part $10*3^{n}$ Normally, we would let $A_{n}=10c3^{n}$, however, in this case the base ($3$) is equal to the constant multiplier in the homogeneous portion ($3$), and so we use the formula provided in the notes that says when $c=b$ is true, then $nb^{n}$ is the solution. So the solution to the non-homogeneous part is $10n3^{n}$
      \item Now, we put it together.
      \item $A_{n}=A_{0}3^{n}+10n3^{n}$
      \item [*] In order to prove that this ($A_{0}3^{n}+10n3^{n}$) is a solution to the nonhomogeneous recurrence relation ($A_{n}=3A_{n-1}+10*3^{n}$), we substitute it (into $3A_{n-1}+10*3^{n}$) in order to attempt to get an identity ($A_{0}3^{n}+10n3^{n}$)
      \item [*] $3(A_{0}3^{n-1}+10(n-1)3^{n-1})+10*3^{n}$
      \item [*] $A_{0}3^{n}+10(n-1)3^{n}+10*3^{n}$
      \item [*] $A_{0}3^{n}+10*3^{n}((n-1)+1)$
      \item [*] $A_{0}3^{n}+10*3^{n}(n-1+1)$
      \item [*] $A_{0}3^{n}+10*3^{n}n$
      \item [*] $A_{0}3^{n}+10n3^{n}$
      \item [*] Since we have the identity $A_{0}3^{n}+10n3^{n}$, $A_{0}3^{n}+10n3^{n}$ is a solution to $A_{n}=3A_{n-1}+10*3^{n}$
      \end{itemize} %}}}
    \item Now that we have solved each piece, we put it all together.
    \item Her method
      \begin{itemize} %{{{
      \item $3^{n}A_{0} -2^{n}\left(\frac{\left(\frac{3}{-2}\right)^{n}-1}{\left(\frac{3}{-2}\right)-1}\right) +10n3^{n}$
      \item [*] In order to prove that this $$3^{n}A_{0} -2^{n}\left(\frac{\left(\frac{3}{-2}\right)^{n}-1}{\left(\frac{3}{-2}\right)-1}\right) +10n3^{n}$$ is a solution to the nonhomogeneous recurrence relation ($A_{n}=3A_{n-1}-2^{n}+10*3^{n}$), we substitute it (into $3A_{n-1}-2^{n}+10*3^{n}$) in order to attempt to get an identity $$3^{n}A_{0} -2^{n}\left(\frac{\left(\frac{3}{-2}\right)^{n}-1}{\left(\frac{3}{-2}\right)-1}\right) +10n3^{n}$$
      \item [*] $$3\left(3^{n-1}A_{0}-2^{n-1}\left(\frac{\left(\frac{3}{-2}\right)^{n-1}-1}{\left(\frac{3}{-2}\right)-1}\right)+10(n-1)3^{n-1}\right)-2^{n}+10*3^{n}$$
      \item [*] $$3^{n}A_{0}-3*2^{n-1}\left(\frac{\left(\frac{3}{-2}\right)^{n-1}-1}{\left(\frac{3}{-2}\right)-1}\right)+10(n-1)3^{n}-2^{n}+10*3^{n}$$
      \item [*] $$3^{n}A_{0}-3*2^{n-1}\left(\frac{\left(\frac{3}{-2}\right)^{n-1}-1}{\left(\frac{3}{-2}\right)-1}\right)+10n3^{n}-10*3^{n}-2^{n}+10*3^{n}$$
      \item [*] $3^{n}A_{0}-3*2^{n-1}\left(\frac{\left(\frac{3}{-2}\right)^{n-1}-1}{\left(\frac{3}{-2}\right)-1}\right)+10n3^{n}-2^{n}$
      \item [*] $3^{n}A_{0} -3*2^{n-1}\left(\frac{(\frac{3}{-2})^{n-1}-1}{(\frac{5}{-2})}\right)-2^{n}+10n3^{n}$
      \item [*] $3^{n}A_{0} -3*2^{n-1}\left(\frac{\frac{3^{n-1}-2^{n-1}}{-2^{n-1} }}{\frac{5}{-2}}\right)-2^{n}+10n3^{n}$
      \item [*] $3^{n}A_{0} -3*2^{n-1}\left(\frac{(3^{n-1}-2^{n-1})*(-2)}{-2^{n-1}*(5) }\right)-2^{n}+10n3^{n}$
      \item [*] $3^{n}A_{0} +3*\left(\frac{(3^{n-1}-2^{n-1})*(-2)}{5}\right)-2^{n}+10n3^{n}$
      \item [*] $3^{n}A_{0} +3*\left(\frac{-2*3^{n-1}+2^{n}}{5}\right)-2^{n}+10n3^{n}$
      \item [*] $3^{n}A_{0} +\frac{-2*3^{n}}{5} +\frac{3*2^{n}}{5}-2^{n}+10n3^{n}$
      \item [*] $3^{n}A_{0} +\frac{-2*3^{n}}{5} +\frac{3}{5}*2^{n}-2^{n}+10n3^{n}$
      \item [*] $3^{n}A_{0} +\frac{-2*3^{n}}{5} -\frac{2}{5}*2^{n}+10n3^{n}$
      \item [*] $3^{n}A_{0} -\frac{2}{5}*3^{n} -\frac{2}{5}*2^{n}+10n3^{n}$
      \item [*] $3^{n}A_{0} -\frac{2}{5}\left(3^{n}+2^{n}\right)+10n3^{n}$
      \item [*] $3^{n}A_{0} -2^{n}(-\frac{2}{5})\left(\frac{3^{n}+2^{n}}{-2^{n}}\right)+10n3^{n}$
      \item [*] $3^{n}A_{0} -2^{n}(-\frac{2}{5})\left(\frac{3^{n}}{-2^{n}}-1\right)+10n3^{n}$
      \item [*] $3^{n}A_{0} -2^{n}(-\frac{2}{5})\left(\left(\frac{3}{-2}\right)^{n}-1\right)+10n3^{n}$
      \item [*] $3^{n}A_{0} -2^{n}\left(\frac{\left(\frac{3}{-2}\right)^{n}-1}{\frac{5}{-2}}\right)+10n3^{n}$
      \item [*] $3^{n}A_{0} -2^{n}\left(\frac{\left(\frac{3}{-2}\right)^{n}-1}{\left(\frac{3}{-2}\right)-1}\right)+10n3^{n}$
      \item [*] Since we have the identity $$3^{n}A_{0} -2^{n}\left(\frac{\left(\frac{3}{-2}\right)^{n}-1}{\left(\frac{3}{-2}\right)-1}\right)+10n3^{n}$$, $$3^{n}A_{0} -2^{n}\left(\frac{\left(\frac{3}{-2}\right)^{n}-1}{\left(\frac{3}{-2}\right)-1}\right)+10n3^{n}$$ is a solution to $A_{n}=3A_{n-1}-2^{n}+10*3^{n}$
      \end{itemize} %}}}
    \item My Method
      \begin{itemize} %{{{
      \item $A_{n}=A_{0}3^{n}+2^{n+1}+10n3^{n}$
      \item [*] In order to prove that this ($A_{0}3^{n}+2^{n+1}+10n3^{n}$) is a solution to the nonhomogeneous recurrence relation ($A_{n}=3A_{n-1}-2^{n}+10*3^{n}$), we substitute it (into $3A_{n-1}-2^{n}+10*3^{n}$) in order to attempt to get an identity ($A_{0}3^{n}+2^{n+1}+10n3^{n}$)
      \item [*] $3(A_{0}3^{n-1}+2^{(n+1)-1}+10(n-1)3^{n-1})-2^{n}+10*3^{n}$
      \item [*] $3(A_{0}3^{n-1}+2^{n+1-1}+10(n-1)3^{n-1})-2^{n}+10*3^{n}$
      \item [*] $3(A_{0}3^{n-1}+2^{n}+10(n-1)3^{n-1})-2^{n}+10*3^{n}$
      \item [*] $A_{0}3^{n}+3*2^{n}+10(n-1)3^{n}-2^{n}+10*3^{n}$
      \item [*] $A_{0}3^{n}+2*2^{n}+10(n-1)3^{n}+10*3^{n}$
      \item [*] $A_{0}3^{n}+2^{n+1}+10(n-1)3^{n}+10*3^{n}$
      \item [*] $A_{0}3^{n}+2^{n+1}+10*3^{n}((n-1)+1)$
      \item [*] $A_{0}3^{n}+2^{n+1}+10*3^{n}(n-1+1)$
      \item [*] $A_{0}3^{n}+2^{n+1}+10*3^{n}(n)$
      \item [*] $A_{0}3^{n}+2^{n+1}+10n3^{n}$
      \item [*] Since we have the identity $A_{0}3^{n}+2^{n+1}+10n3^{n}$, $A_{0}3^{n}+2^{n+1}+10n3^{n}$ is a solution to $A_{n}=3A_{n-1}-2^{n}+10*3^{n}$
      \end{itemize} %}}}
    \end{itemize}
  \item $B_{n}=B_{n-1}+cn$, where c is a nonzero constant.
    \begin{itemize}
    \item The formula is $$F_{n}=c^{n}F_{0} + \sum\limits_{i=0}^{n-1}c^{i}f(n-i)$$
    \item In this problem, $c=1$ and $f(n)=cn$
    \item Plugging in to the formula we get $$B_{n}=1^{n}B_{0} + \sum\limits_{i=0}^{n-1}1^{i}c(n-i)$$
    \item $$B_{n}=B_{0} + \sum\limits_{i=0}^{n-1}c(n-i)$$
    \item $$B_{n}=B_{0} + \sum\limits_{i=0}^{n-1}c(n-i)$$
    \item $$B_{n}=B_{0} + c\sum\limits_{i=0}^{n-1}(n-i)$$
    \item $$B_{n}=B_{0} + c\sum\limits_{k=1}^{k}(k)$$
    \item $B_{n}=B_{0} + c\left(\frac{n^{2}+n}{2}\right)$
    \item $B_{n}=B_{0} + \left(\frac{c\left(n^{2}+n\right)}{2}\right)$
    \item $B_{n}=B_{0} + \left(\frac{cn\left(n+1\right)}{2}\right)$
    \item [*] In order to prove that this $$B_{n}=B_{0} + \left(\frac{cn\left(n+1\right)}{2}\right)$$ is a solution to the nonhomogeneous recurrence relation ($B_{n}=B_{n-1}+cn$), we substitute it (into $B_{n}=B_{n-1}+cn$) in order to attempt to get an identity $$B_{n}=B_{0} + \left(\frac{cn\left(n+1\right)}{2}\right)$$
    \item [*] $\left(B_{0}+\left(\frac{c(n-1)\left((n-1)+1\right)}{2}\right)\right)+cn$
    \item [*] $B_{0}+\left(\frac{c(n-1)n}{2}\right)+cn$
    \item [*] $B_{0}+\frac{cn(n-1)}{2}+cn$
    \item [*] $B_{0}+\frac{1}{2}(n-1)cn+cn$
    \item [*] $B_{0}+\frac{1}{2}(n-1)cn+cn$
    \item [*] $B_{0}+\frac{1}{2}cn^{2}-\frac{1}{2}cn+cn$
    \item [*] $B_{0}+\frac{1}{2}cn^{2}+\frac{1}{2}cn$
    \item [*] $B_{0}+\frac{1}{2}(cn(n+1))$
    \item [*] $B_{0}+\left(\frac{cn(n+1)}{2}\right)$
    \end{itemize}
  \end{enumerate}
%\item Consider the following recurrence relation, which will be used in analyzing the merge-sort algorithm: \\
%  $S(0)=1$; $S(1)=1$; $S(n)=S(\lceil \frac{n}{2} \rceil)+1$ for all $n>0$
%  \begin{enumerate}
%  \item Find the exact solution to the given recurrence relation, i.e., find a function $f(n)$ such that $S(n) = f(n)$ for all $n \geq 0$. You need to either show your process of finding such a solution or provide function $f(n)$ and prove (by strong induction) that it is a solution to the given recurrence relation. You can use Theorem 2.1.5 (page 12 in the textbook) in your proof, but you need to cite which part of that theorem you are using.
%    \begin{itemize}
%    \item First, we will begin by using theorem 2.1.5c that says that $\lfloor\frac{n+1}{2}\rfloor=\lceil\frac{n}{2}\rceil$
%    \item $S(n)=S(\lfloor\frac{n+1}{2}\rfloor)+1$
%    \item In order to attempt to solve, we will use forward substitution.
%    \begin{equation}
%    S(n)=S(\lfloor\frac{n+1}{2^{2}}\rfloor)+2
%    \end{equation}
%    \begin{equation}
%    S(n)=S(\lfloor\frac{n+1}{2^{3}}\rfloor)+3
%    \end{equation}
%    \begin{equation}
%    S(n)=S(\lfloor\frac{n+1}{2^{k+1}}\rfloor)+k+1
%    \end{equation}
%    \item Theorum 2.28b in the notes states that $k=\lfloor\log{(n)}\rfloor$
%    \item $S(n)=S(\lfloor\frac{n+1}{2^{\lfloor\log{(n)}\rfloor+1}}\rfloor)+\lfloor\log{(n)}\rfloor+1$
%    \item $$\sum\limits_{i=0}^{\lfloor\log{(n)}\rfloor}i + \sum\limits_{i=0}^{\lfloor\log{(n)}\rfloor}\lfloor\frac{n+1}{2^{i}}\rfloor$$
%    \item Now is the induction step. Let k be an arbitrary integer. Assume that $$\sum\limits_{i=0}^{\lfloor\log{(k)}\rfloor}i + \sum\limits_{i=0}^{\lfloor\log{(k)}\rfloor}\lfloor\frac{k+1}{2^{i}}\rfloor$$ is a solution to $S(k)=S(\lceil \frac{k}{2} \rceil)+1$. We will prove that $$\sum\limits_{i=0}^{\lfloor\log{(k+1)}\rfloor}i + \sum\limits_{i=0}^{\lfloor\log{(k+1)}\rfloor}\lfloor\frac{k+1+1}{2^{i}}\rfloor$$ is a solution to $S(k+1)=S(\lceil \frac{k+1}{2} \rceil)+1$
%    \item We have $$\sum\limits_{i=0}^{\lfloor\log{(k)}\rfloor}i + (k+1) + \sum\limits_{i=0}^{\lfloor\log{(k)}\rfloor}\lfloor\frac{k+1}{2^{i}}\rfloor +\lfloor\frac{k+1}{2^{k}}\rfloor$$ is a solution to $S(k)=S(\lceil \frac{k}{2} \rceil)+1$
%    \item By the induction assumption we have $$\sum\limits_{i=0}^{\lfloor\log{(k)}\rfloor}i +(k+1)+ \sum\limits_{i=0}^{\lfloor\log{(k)}\rfloor}\lfloor\frac{k+1}{2^{i}}\rfloor +\lfloor\frac{k+1}{2^{k+1}}\rfloor$$
%    \item $$\sum\limits_{i=0}^{\lfloor\log{(k+1)}\rfloor}i + \sum\limits_{i=0}^{\lfloor\log{(k)}\rfloor}\lfloor\frac{k+1}{2^{i}}\rfloor +\lfloor\frac{k+1}{2^{k+1}}\rfloor$$
%    \item $$\sum\limits_{i=0}^{\lfloor\log{(k+1)}\rfloor}i + \sum\limits_{i=0}^{\lfloor\log{(k+1)}\rfloor}\lfloor\frac{k+1}{2^{i}}\rfloor$$
%    \item This proves that $$\sum\limits_{i=0}^{\lfloor\log{(k+1)}\rfloor}i + \sum\limits_{i=0}^{\lfloor\log{(k+1)}\rfloor}\lfloor\frac{k+1}{2^{i}}\rfloor$$ is a solution to $S(k)=S(\lceil\frac{k}{2}\rceil)+1$
%    \end{itemize}
%  \item Find an Big-Theta formula for the solution to the given recurrence relation, i.e., find a function $g(n)$ such that $S(n) = \theta(g(n))$. Justify your answer.
%      \begin{itemize}
%      \item From the slides, $$\sum\limits_{i=0}{\lfloor\log{(n)}\rfloor}\frac{n}{2^i}=\theta(n\log{(n)})$$
%      \item As the other parts Big O is n, the Big O of this is nlog(n).
%      \end{itemize}
%  \end{enumerate}
%\item Find a Big-Theta formula for the solution to the following recurrence relation: \\
%  $T(0)=a$; $T(n)=T(\frac{n}{2})+T(\frac{n}{2})+cn$ for all $n>0$ where a and c are nonzero constants. You may use Table 4.3, page 149 in the textbook, but need to justify your answer.
\end{enumerate}
\end{document}
