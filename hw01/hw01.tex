% vim: nu expandtab shiftwidth=2 softtabstop=2 foldmethod=marker

\documentclass{article}
\usepackage{fancyhdr}
\usepackage[margin=0.5in]{geometry}
\def\mydate{\leavevmode\hbox{\the\year-\twodigits\month-\twodigits\day}}
\def\twodigits#1{\ifnum#1<10 0\fi\the#1}
\title{Homework 1}
\date{\mydate}
\author{George Sparrow}

\fancypagestyle{plain}
{
  \fancyhf{}
  \lfoot{\author{George Sparrow}}
  \cfoot{\thepage}
  \rfoot{\date{\mydate}}
  \renewcommand{\headrulewidth}{0pt}
}
\pagestyle{plain}

\begin{document}
\maketitle
\newpage

\begin{enumerate}
\item Prove by induction that for any positive integer n, \\
  {\Large $\sum\limits_{i=1}^{n}i^{3}=\frac{n^{2}(n+1)^{2}}{4}$}
    \begin{itemize}
    \item [] First we must prove the base case, in this problem, since n must be a positive integer, the base case is where n is equal to 1.
    \item [] {\Large $\sum\limits_{i=1}^{1}i^{3}=\frac{n^{2}(n+1)^{2}}{4}$}
    \item [] {\Large $1^{3}=\frac{1^{2}(1+1)^{2}}{4}$}
    \item [] {\Large $1=\frac{1(1+1)^{2}}{4}$}
    \item [] {\Large $1=\frac{1(2)^{2}}{4}$}
    \item [] {\Large $1=\frac{1*4}{4}$}
    \item [] {\Large $1=\frac{4}{4}$}
    \item [] $1=1$
    \item Since 1 is equal to 1, we proved that the base case is true.
    \item Now is the induction step. Let k be an arbitrary integer greater than 1. Assume {\Large $\sum\limits_{i=1}^{k}i^{3}=\frac{k^{2}(k+1)^{2}}{4}$} We will prove that {\Large $\sum\limits_{i=1}^{k+1}i^{3}=\frac{(k+1)^{2}((k+1)+1)^{2}}{4}$}
    \item We have {\Large $\sum\limits_{i=1}^{k+1}i^{3}=\left( \sum\limits_{i=1}^{k}i^{3} \right) + (k+1)^{3}$}
    \item By the inductive assumption we have {\Large $\sum\limits_{i=1}^{k+1}i^{3}=\left(\frac{k^{2}(k+1)^{2}}{4} \right) + (k+1)^{3}$}
    \item {\Large $\sum\limits_{i=1}^{k+1}i^{3}=\left(\frac{k^{2}(k^{2}+2k+1)}{4} \right) + (k+1)^{3}$}
    \item {\Large $\sum\limits_{i=1}^{k+1}i^{3}=\left(\frac{k^{4}+2k^{3}+k^{2}}{4} \right) + (k+1)^{3}$}
    \item {\Large $\sum\limits_{i=1}^{k+1}i^{3}=\left(\frac{k^{4}+2k^{3}+k^{2}}{4} \right) + (k^{2}+2k+1)(k+1)$}
    \item {\Large $\sum\limits_{i=1}^{k+1}i^{3}=\left(\frac{k^{4}+2k^{3}+k^{2}}{4} \right) + k^{3}+3k^{2}+3k+1$}
    \item {\Large $\sum\limits_{i=1}^{k+1}i^{3}=\frac{k^{4}+2k^{3}+k^{2}}{4} + \frac{4k^{3}+12k^{2}+12k+4}{4}$}
    \item {\Large $\sum\limits_{i=1}^{k+1}i^{3}=\frac{k^{4}+6k^{3}+13k^{2}+12k+4}{4}$}
    \item {\Large $\sum\limits_{i=1}^{k+1}i^{3}=\frac{(k+1)^{2}(k+2)^{2}}{4}$}
    \item {\Large $\sum\limits_{i=1}^{k+1}i^{3}=\frac{(k+1)^{2}((k+1)1)^{2}}{4}$}
    \item This proves that {\Large $\sum\limits_{i=1}^{k+1}i^{3}=\frac{(k+1)^{2}((k+1)+1)^{2}}{4}$} holds. By induction, {\Large $\sum\limits_{i=1}^{n}i^{3}=\frac{n^{2}(n+1)^{2}}{4}$} holds for all positive integers.
    \end{itemize}

\item In each case below, explain why the given expression is true:
  \begin{enumerate}
  \item $3n^{2} + 100n + \log{(n)} = O(n^{2})$
    \begin{itemize}
    \item [] First we use the limit definition of Big O.
    \item {\Large $$\lim_{n \to \infty} \frac{3n^{2} +100n +\log{(n)}}{n^{2}} < \infty$$}
    \item [] Now we distribute the denominator.
    \item {\Large $$\lim_{n \to \infty} \frac{3n^{2}}{n^{2}} + \lim_{n \to \infty} \frac{100n}{n^{2}} + \lim_{n \to \infty} \frac{\log{(n)}}{n^{2}} < \infty$$}
    \item [] Now we actually divide these.
    \item {\Large $$\lim_{n \to \infty} 3 + \lim_{n \to \infty} \frac{100}{n} + \lim_{n \to \infty} \frac{\log{(n)}}{n^{2}} < \infty$$}
    \item [] Now we use the limit definion that says {\Large $$\lim_{n \to \infty} \frac{c}{n^{a}} =0$$} where c and a are positive.
    \item {\Large $$\lim_{n \to \infty} 3 +0+ \lim_{n \to \infty} \frac{\log{(n)}}{n^{2}} < \infty$$}
    \item {\Large $$\lim_{n \to \infty} 3 + \lim_{n \to \infty} \frac{\log{(n)}}{n^{2}} < \infty$$}
    \item [] Now we use the limit definition that says $$\lim_{n \to \infty} c = c$$ where c is some constant.
    \item {\Large $$3 + \lim_{n \to \infty} \frac{\log{(n)}}{n^{2}} < \infty$$}
    \item [] Since we have a different base, we need to use the change of base formula for logarithms. $log_{b}x =ln(x)/ln(b)$
    \item {\Large $$3 + \lim_{n \to \infty} \frac{\ln{(n)}}{\ln{2}*n^{2}} < \infty$$}
    \item [] Now, since $$\lim{n \to \infty} \ln{n} = \infty$$, and $$\lim_{n \to \infty} \ln{2}*n^{2} = \infty$$ we use lopitals rule and take the derivative. 
    \item {\Large $$3 + \lim_{n \to \infty} \frac{1}{2\ln{2}*n^{2}} < \infty$$}
    \item [] Now we use the limit definion that says {\Large $$\lim_{n \to \infty} \frac{c}{n^{a}} =0$$} where c and a are positive.
    \item {\Large $$3 + 0 < \infty$$}
    \item {\Large $$3 < \infty$$}
    \item As 3 is less than infinity, we have proved that $3n^{2} + 100n + \log{(n)} = O(n^{2})$
    \end{itemize}
  \item $(\sqrt{n} +1)^{8} = O(n^{4})$
    \begin{itemize}
    \item [] Begin by changing the exponent on n
    \item {\Large{$(n^{\frac{1}{2}} +1)=O(n^{4})$}}
    \item [] Then expand the left side.
    \item {\Large{$n^{4} + 8n^{\frac{7}{2}} +28n^{3} +56n^{\frac{5}{2}} + 70n^{2} +56n^{\frac{3}{2}} +28n +8n^{\frac{1}{2}} +1 = O(n^{4})$}}
    \item [] Now we use the limit definition of Big O.
    \item {\Large{$$\lim_{n \to \infty} \frac{n^{4} + 8n^{\frac{7}{2}} +28n^{3} +56n^{\frac{5}{2}} + 70n^{2} +56n^{\frac{3}{2}} +28n +8n^{\frac{1}{2}} +1 = O(n^{4})}{n^{4}} < \infty$$}}
    \item [] Now we distribute the denominator.
    $$\lim_{n \to \infty} \frac{n^{4}}{n^{4}} + \lim_{n \to \infty} \frac{8n^{\frac{7}{2}}}{n^{4}} + \lim_{n \to \infty} \frac{28n^{3}}{n^{4}} + \lim_{n \to \infty} \frac{56n^{\frac{5}{2}}}{n^{4}} + \lim_{n \to \infty} \frac{70n^{2}}{n^{4}} + \lim_{n \to \infty} \frac{56n^{\frac{3}{2}}}{n^{4}} + \lim_{n \to \infty} \frac{n}{n^{4}} + \lim{n \to \infty} \frac{8n^{\frac{1}{8}}}{n^{4}} + \lim_{n \to \infty} \frac{1}{n^{4}} < \infty$$
    \item [] Now we actually divide these.
    $$\lim_{n \to \infty} 1 + \lim_{n \to \infty} \frac{8}{n^{\frac{1}{2}}} + \lim_{n \to \infty} \frac{28}{n} + \lim_{n \to \infty} \frac{56}{n^{\frac{3}{2}}} + \lim_{n \to \infty} \frac{70}{n^{2}} + \lim_{n \to \infty} \frac{56}{n^{\frac{5}{2}}} + \lim_{n \to \infty} \frac{1}{n^{3}} + \lim{n \to \infty} \frac{8}{n^{\frac{7}{2}}} + \lim_{n \to \infty} \frac{1}{n^{4}} < \infty$$
    \item [] Now we use the limit definition that says {\Large $$\lim_{n \to \infty} \frac{c}{n^{a}} =0$$} where c and a are positive.
    \item {\Large{$$\lim_{n \to \infty} 1 +0+0+0+0+0+0+0+0 < \infty$$}}
    \item {\Large{$$\lim_{n \to \infty} 1 < \infty$$}}
    \item [] Now we use the limit definition that says $$\lim_{n \to \infty} c = c$$ where c is some constant.
    \item $1 < \infty$
    \item As 1 is less than infinity, we have proved that $(\sqrt{n} +1)^{8} =O(n^{4})$
    \end{itemize}
  \end{enumerate}

\item Prove by contradiction that $100n+2\neq O(\sqrt{n})$
  \begin{itemize}
  \item Suppose that $100n+2 = O \sqrt{n}$
  \item [] Then we can turn sqare root into a power using its definition.
  \item {\Large $100n+2 = O n^{\frac{1}{2}}$}
  \item [] Now we use the limit definition of Big O.
  \item {\Large $$\lim_{n \to \infty} \frac{100n+2}{n^{\frac{1}{2}}} < \infty$$}
  \item [] Now we distribute the denominator.
  \item {\Large $$\lim_{n \to \infty} \frac{100n}{n^{\frac{1}{2}}} + \lim_{n \to \infty} \frac{2}{n^{\frac{1}{2}}} < \infty$$}
  \item [] Now we use the limit definition that says {\Large $$\lim_{n \to \infty} \frac{c}{n^{a}} =0$$} where c and a are positive.
  \item {\Large $$\lim_{n \to \infty} \frac{100n}{n^{\frac{1}{2}}} + 0 < \infty$$}
  \item {\Large $$\lim_{n \to \infty} \frac{100n}{n^{\frac{1}{2}}} < \infty$$}
  \item [] Now we use the limit definition that says {\large $$\lim_{n \to \infty} n = \infty$$}
  \item $\infty < \infty$
  \item As infinity is not less than infinity, we have proved by contradiction that $100n+2 \neq O(\sqrt{n})$
  \end{itemize}
\end{enumerate}
\end{document}
