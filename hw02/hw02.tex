% vim: nu expandtab shiftwidth=2 softtabstop=2 foldmethod=marker

\documentclass{article}
\usepackage{fancyhdr}
\def\mydate{\leavevmode\hbox{\the\year-\twodigits\month-\twodigits\day}}
\def\twodigits#1{\ifnum#1<10 0\fi\the#1}
\title{Homework 2}
\date{\mydate}
\author{George Sparrow}

\fancypagestyle{plain}
{
  \fancyhf{}
  \lfoot{\author{George Sparrow}}
  \cfoot{\thepage}
  \rfoot{\date{\mydate}}
  \renewcommand{\headrulewidth}{0pt}
}
\pagestyle{plain}

\begin{document}
\maketitle
\newpage

\begin{enumerate}
\item For each recurrence relation below, determine if it is linear. If it it linear, determine all the non-zero coefficients and the nonhomogeneous term (if exists) of the recurrence relation.
  \begin{enumerate}
  \item $A_{n}=2A_{n-1}+2^{n}$
    \begin{itemize}
    \item [] All things in the form $$F_{n}=f(n) \sum\limits_{i=1}^{n}c_{i}(n)F_{n-i}$$ are linear recurrrence relations.
    \item In this case, $f(n)=2^{n}$, and $c=2$, so $$F_{n}=2^{n} \sum\limits_{i=1}^{n}2F_{n-i}$$ thus it is a linear reccurence relation. The non-zero coefficient is $2$ and the nonhomogenous term is $2^{n}$
    \end{itemize}
  \item $A_{n}=\log{(n)}A_{n-2}+3A_{n-3}$
    \begin{itemize}
    \item In this case it is a linear recurrence relation and the non-zero coefficients are $\log{(n)}$ and $3$, and there is no nonhomogenous term.
    \end{itemize}
  \item {\Large $A_{n}=\frac{A_{n-1}}{A_{n-2}}+1$}
    \begin{itemize}
    \item In this case, it is not a linear recurrence relation because they previous terms are dividing each other. The nonhomogenous term is $1$
    \end{itemize}
  \end{enumerate}
\item Check that the recurrence relation
  \begin{equation}
  F_{n}=6F_{n-1}-9F_{n-2}
  \end{equation}
has the following solution:
  \begin{equation}
  F_{n}=(\alpha_{1}+\alpha_{2}n)3^{n}
  \end{equation}
where $\alpha_{1}$ and $\alpha_{2}$ are any constants.
  \begin{itemize}
  \item [] In order to prove that this (equation 2) is a solution to the recurrence relation (equation 1), we substitute it (into equation 1) in order to attempt to get an identity (equation 2).
  \item $6((\alpha_{1}+\alpha_{2}(n-1))3^{(n-1)}) -9((\alpha_{1}+\alpha_{2}(n-2))3^{(n-2)})$
  \item [$\Leftrightarrow$]$2((\alpha_{1}+\alpha_{2}(n-1))3^{(n)}) -((\alpha_{1}+\alpha_{2}(n-2))3^{(n)})$
  \item [$\Leftrightarrow$]$3^{n}(2(\alpha_{1}+\alpha_{2}(n-1)) -(\alpha_{1}+\alpha_{2}(n-2)))$
  \item [$\Leftrightarrow$]$3^{n}(2\alpha_{1}+2n\alpha_{2}-2\alpha_{2} -\alpha_{1}-n\alpha_{2}+2\alpha_{2})$
  \item [$\Leftrightarrow$]$3^{n}(\alpha_{1}+n\alpha_{2})$
  \item [] through the commutative property of multiplication, which says that $ab=ba$, we get
  \item $(\alpha_{1}+\alpha_{2}n)3^{n}$
  \item As this is equation 2, we have the identity which proves that equation 2 is a solution to equation 1.
  \end{itemize}

\item Find the solution to each recurrence relation below.
  \begin{enumerate}
  \item $A_{0}=-2$; $A_{n}=3A_{n-1}$ for all $n\geq1$.
    \begin{itemize}
    \item [] Using the general solution for all linear homogenous first order recurrence relations, $F_{n}=A_{0}*c^{n}$ we get
    \item $A_{n}=(-2)*3^{n}$
    \item [] In order to prove that this ($A_{n}=-2*3^{n}$) is a solution to the recurrence relation ($3A_{n-1}$), we substitute it (into $3A_{n-1}$) in order to attempt to get an identity.
    \item $3(-2*3^{(n-1)})$
    \item [$\Leftrightarrow$]$-2*3*3^{(n-1)})$
    \item [$\Leftrightarrow$]$-2*3^{n})$
    \item As this is $-2*3^{n}$, we have the identity which proves that $2*3^{n}$ is a solution to $3A_{n-1}$
    \end{itemize}
  \item $B_{0}=1$; $B_{1}=8$; $B_{n}=B_{n-1}+2B_{n-2}$ for all $n\geq2$
    \begin{itemize}
    \item [] First, we need to place it in the form $Ax^{2}-Bx-C$, the characteristic equation.
    \item $x^{2}-x-2$
    \item [] Now we use the formula {\Large $x=\frac{-b \pm \sqrt{b^{2}-4ac}}{2a}$}
    \item {\Large $x=\frac{1 \pm \sqrt{1-4(1)(-2)}}{2}$}
    \item [$\Leftrightarrow$]{\Large $x=\frac{1 \pm \sqrt{1+8}}{2}$}
    \item [$\Leftrightarrow$]{\Large $x=\frac{1 \pm \sqrt{9}}{2}$}
    \item [$\Leftrightarrow$]{\Large $x=\frac{1 \pm 3}{2}$}
    \item {\Large $x_{1}=\frac{1+3}{2}$}
    \item {\Large $x_{1}=\frac{4}{2}$}
    \item {\Large $x_{1}=2$}
    \item {\Large $x_{2}=\frac{1-3}{2}$}
    \item {\Large $x_{2}=\frac{-2}{2}$}
    \item {\Large $x_{2}=-1$}
    \item [] Since $x_{1}$ and $x_{2}$ are not equal, we use the formula \\
      $F_{n}=\alpha_{1}(x_{1})^{n}+\alpha_{2}(x_{2})^{n} $ for all $n\geq0$
    \item [] Now we create a system of equations to solve for $\alpha_{1}$ and $\alpha_{2}$
    \item $B_{0}=\alpha_{1}+\alpha_{2}=1$
    \item [] $B_{1}=\alpha_{1}x_{1}+2\alpha_{2}x_{2}=8$
    \item [] Now we can use substitution to solve for $\alpha_{1}$ and $\alpha_{2}$
    \item [] $\alpha_{1}=1-\alpha_{2}$
    \item [] $(1-\alpha_{2})x_{1} +2\alpha_{2}x_{2}=8$
    \item [$\Leftrightarrow$] $(1-\alpha_{2})(2) +2\alpha_{2}(-1)=8$
    \item [$\Leftrightarrow$] $2-2\alpha_{2} -2\alpha_{2}=8$
    \item [$\Leftrightarrow$] $2-4\alpha_{2}=8$
    \item [$\Leftrightarrow$] $-4\alpha_{2}=6$
    \item [$\Leftrightarrow$] $\alpha_{2}=-\frac{3}{2}$
    \item [] $\alpha_{1}=1-\alpha_{2}$
    \item [] $\alpha_{1}=1+\frac{3}{2}$
    \item [] $\alpha_{1}=\frac{5}{2}$
    \item $F_{n}=\frac{5}{2}(2)^{n}-\frac{3}{2}(-1)^{n} $ for all $n\geq0$
    \item [] In order to prove that this ($F_{n}=\frac{5}{2}(2)^{n}-\frac{3}{2}(-1)^{n}$) is a solution to the recurrence relation ($B_{n-1}+2B_{n-2}$), we substitute it into \\($B_{n-1}+2B_{n-2}$) in order to attempt to get an identity \\($F_{n}=\frac{5}{2}(2)^{n}-\frac{3}{2}(-1)^{n}$).
    \item [] $B_{n-1}+2B_{n-2}$
    \item [$\Leftrightarrow$] $(\frac{5}{2}(2)^{n-1}-\frac{3}{2}(-1)^{n-1})+2(\frac{5}{2}(2)^{n-2}-\frac{3}{2}(-1)^{n-2})$
    \item [$\Leftrightarrow$] $\frac{5}{2}(2)^{n-1}+\frac{3}{2}(-1)^{n})+2\frac{5}{2}(2)^{n-2}-2\frac{3}{2}(-1)^{n})$
    \item [$\Leftrightarrow$] $\frac{5}{2}(2)^{n-1}-\frac{3}{2}(-1)^{n})+2\frac{5}{2}(2)^{n-2}$
    \item [$\Leftrightarrow$] $\frac{5}{2}(2)^{n-1}-\frac{3}{2}(-1)^{n})+\frac{5}{2}(2)^{n-1}$
    \item [$\Leftrightarrow$] $2\frac{5}{2}(2)^{n-1}-\frac{3}{2}(-1)^{n})$
    \item [$\Leftrightarrow$] $\frac{5}{2}(2)^{n}-\frac{3}{2}(-1)^{n})$
    \item As this is $\frac{5}{2}(2)^{n}-\frac{3}{2}(-1)^{n}$, we have the identity which proves that $\frac{5}{2}(2)^{n}-\frac{3}{2}(-1)^{n}$ is a solution to $F_{n}=B_{n-1}+2B_{n-2}$
    \end{itemize}
  \end{enumerate}
\end{enumerate}
\end{document}
